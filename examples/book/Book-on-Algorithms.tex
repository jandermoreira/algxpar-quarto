% Options for packages loaded elsewhere
\PassOptionsToPackage{unicode}{hyperref}
\PassOptionsToPackage{hyphens}{url}
%
\documentclass[
  letterpaper,
]{scrbook}

\usepackage{amsmath,amssymb}
\usepackage{iftex}
\ifPDFTeX
  \usepackage[T1]{fontenc}
  \usepackage[utf8]{inputenc}
  \usepackage{textcomp} % provide euro and other symbols
\else % if luatex or xetex
  \usepackage{unicode-math}
  \defaultfontfeatures{Scale=MatchLowercase}
  \defaultfontfeatures[\rmfamily]{Ligatures=TeX,Scale=1}
\fi
\usepackage{lmodern}
\ifPDFTeX\else  
    % xetex/luatex font selection
\fi
% Use upquote if available, for straight quotes in verbatim environments
\IfFileExists{upquote.sty}{\usepackage{upquote}}{}
\IfFileExists{microtype.sty}{% use microtype if available
  \usepackage[]{microtype}
  \UseMicrotypeSet[protrusion]{basicmath} % disable protrusion for tt fonts
}{}
\makeatletter
\@ifundefined{KOMAClassName}{% if non-KOMA class
  \IfFileExists{parskip.sty}{%
    \usepackage{parskip}
  }{% else
    \setlength{\parindent}{0pt}
    \setlength{\parskip}{6pt plus 2pt minus 1pt}}
}{% if KOMA class
  \KOMAoptions{parskip=half}}
\makeatother
\usepackage{xcolor}
\setlength{\emergencystretch}{3em} % prevent overfull lines
\setcounter{secnumdepth}{5}
% Make \paragraph and \subparagraph free-standing
\ifx\paragraph\undefined\else
  \let\oldparagraph\paragraph
  \renewcommand{\paragraph}[1]{\oldparagraph{#1}\mbox{}}
\fi
\ifx\subparagraph\undefined\else
  \let\oldsubparagraph\subparagraph
  \renewcommand{\subparagraph}[1]{\oldsubparagraph{#1}\mbox{}}
\fi


\providecommand{\tightlist}{%
  \setlength{\itemsep}{0pt}\setlength{\parskip}{0pt}}\usepackage{longtable,booktabs,array}
\usepackage{calc} % for calculating minipage widths
% Correct order of tables after \paragraph or \subparagraph
\usepackage{etoolbox}
\makeatletter
\patchcmd\longtable{\par}{\if@noskipsec\mbox{}\fi\par}{}{}
\makeatother
% Allow footnotes in longtable head/foot
\IfFileExists{footnotehyper.sty}{\usepackage{footnotehyper}}{\usepackage{footnote}}
\makesavenoteenv{longtable}
\usepackage{graphicx}
\makeatletter
\def\maxwidth{\ifdim\Gin@nat@width>\linewidth\linewidth\else\Gin@nat@width\fi}
\def\maxheight{\ifdim\Gin@nat@height>\textheight\textheight\else\Gin@nat@height\fi}
\makeatother
% Scale images if necessary, so that they will not overflow the page
% margins by default, and it is still possible to overwrite the defaults
% using explicit options in \includegraphics[width, height, ...]{}
\setkeys{Gin}{width=\maxwidth,height=\maxheight,keepaspectratio}
% Set default figure placement to htbp
\makeatletter
\def\fps@figure{htbp}
\makeatother

\makeatletter
\@ifpackageloaded{tcolorbox}{}{\usepackage[skins,breakable]{tcolorbox}}
\@ifpackageloaded{fontawesome5}{}{\usepackage{fontawesome5}}
\definecolor{quarto-callout-color}{HTML}{909090}
\definecolor{quarto-callout-note-color}{HTML}{0758E5}
\definecolor{quarto-callout-important-color}{HTML}{CC1914}
\definecolor{quarto-callout-warning-color}{HTML}{EB9113}
\definecolor{quarto-callout-tip-color}{HTML}{00A047}
\definecolor{quarto-callout-caution-color}{HTML}{FC5300}
\definecolor{quarto-callout-color-frame}{HTML}{acacac}
\definecolor{quarto-callout-note-color-frame}{HTML}{4582ec}
\definecolor{quarto-callout-important-color-frame}{HTML}{d9534f}
\definecolor{quarto-callout-warning-color-frame}{HTML}{f0ad4e}
\definecolor{quarto-callout-tip-color-frame}{HTML}{02b875}
\definecolor{quarto-callout-caution-color-frame}{HTML}{fd7e14}
\makeatother
\makeatletter
\@ifpackageloaded{bookmark}{}{\usepackage{bookmark}}
\makeatother
\makeatletter
\@ifpackageloaded{caption}{}{\usepackage{caption}}
\AtBeginDocument{%
\ifdefined\contentsname
  \renewcommand*\contentsname{Table of contents}
\else
  \newcommand\contentsname{Table of contents}
\fi
\ifdefined\listfigurename
  \renewcommand*\listfigurename{List of Figures}
\else
  \newcommand\listfigurename{List of Figures}
\fi
\ifdefined\listtablename
  \renewcommand*\listtablename{List of Tables}
\else
  \newcommand\listtablename{List of Tables}
\fi
\ifdefined\figurename
  \renewcommand*\figurename{Figure}
\else
  \newcommand\figurename{Figure}
\fi
\ifdefined\tablename
  \renewcommand*\tablename{Table}
\else
  \newcommand\tablename{Table}
\fi
}
\@ifpackageloaded{float}{}{\usepackage{float}}
\floatstyle{ruled}
\@ifundefined{c@chapter}{\newfloat{codelisting}{h}{lop}}{\newfloat{codelisting}{h}{lop}[chapter]}
\floatname{codelisting}{Listing}
\newcommand*\listoflistings{\listof{codelisting}{List of Listings}}
\makeatother
\makeatletter
\makeatother
\makeatletter
\@ifpackageloaded{caption}{}{\usepackage{caption}}
\@ifpackageloaded{subcaption}{}{\usepackage{subcaption}}
\makeatother
\makeatletter
\@ifpackageloaded{float}{}{\usepackage{float}}
\makeatother
\makeatletter
\@ifpackageloaded{algorithm}{}{\usepackage{algorithm}}
\makeatother
\makeatletter
\@ifpackageloaded{algxpar}{}{\usepackage[brazilian]{algxpar}}
\makeatother
\ifLuaTeX
  \usepackage{selnolig}  % disable illegal ligatures
\fi
\usepackage{bookmark}

\IfFileExists{xurl.sty}{\usepackage{xurl}}{} % add URL line breaks if available
\urlstyle{same} % disable monospaced font for URLs
\hypersetup{
  pdftitle={Book on Algorithms},
  pdfauthor={Jander Moreira},
  hidelinks,
  pdfcreator={LaTeX via pandoc}}

\title{Book on Algorithms}
\author{Jander Moreira}
\date{2023-09-14}

\begin{document}
\frontmatter
\maketitle

\floatstyle{plaintop}\restylefloat{algorithm}

\counterwithin{algorithm}{chapter}

\renewcommand*\contentsname{Table of contents}
{
\setcounter{tocdepth}{2}
\tableofcontents
}
\mainmatter
\bookmarksetup{startatroot}

\chapter*{Welcome}\label{welcome}
\addcontentsline{toc}{chapter}{Welcome}

\markboth{Welcome}{Welcome}

This is a book with algorithms.

Look for Algoritmo~\ref{alg-novo}.

See Algoritmo~\ref{alg-red} for red and Algoritmo~\ref{alg-blue} for
blue.

\begin{algorithm}
\caption{\label{alg-novo}}
\begingroup%

\begin{algorithmic}
    \State $i \gets 10$
    \If{$i\geq 5$} 
        \State $i \gets i-1$
    \Else
        \If{$i\leq 3$}
            \State $i \gets i+2$
        \EndIf
    \EndIf 
\end{algorithmic}

\endgroup
\end{algorithm}

\bookmarksetup{startatroot}

\chapter{One}\label{one}

\begin{tcolorbox}[enhanced jigsaw, colback=white, leftrule=.75mm, rightrule=.15mm, bottomrule=.15mm, arc=.35mm, colframe=quarto-callout-color-frame, breakable, toprule=.15mm, opacityback=0, left=2mm]

\begin{tcolorbox}[enhanced jigsaw, colback=white, coltitle=black, opacitybacktitle=0.6, arc=.35mm, title=\textcolor{quarto-callout-tip-color}{\faLightbulb}\hspace{0.5em}{Tip}, breakable, opacityback=0, titlerule=0mm, leftrule=.75mm, rightrule=.15mm, toptitle=1mm, colframe=quarto-callout-tip-color-frame, colbacktitle=quarto-callout-tip-color!10!white, bottomtitle=1mm, bottomrule=.15mm, toprule=.15mm, left=2mm]

\begin{algorithm}[H]
\caption{\label{alg-blue}A new way to do it.}
\begingroup%

%| pdf-float: false
%| title: "A new way to do it."
%| label: #alg-blue
   
\begin{algorithmic}<language = brazilian, keyword color = blue>
    \Description Criação de um arquivo com informações para manter uma agenda simplificada.
    \Input Uma sequência de contatos, separada nos campos \Id{nome}, \Id{email}, \Id{telefone}, \Id{mês} e \Id{ano}
    \Output Os dados de entrada armazenados em um arquivo
    \Procedure{CrieAgenda}{}[Criação a partir de dados externos]
      \Statep{Crie \Id{arquivo} como um arquivo vazio com acesso de escrita}
      \While{há dados para na entrada}
          \Statep{Obtenha os valores para \Id{nome}, \Id{email}, \Id{telefone}, \Id{mês} e \Id{ano}}
          \Statex
          \Statep{\Write \Id{nome} em \Id{arquivo}}
          \Statep{\Write \Id{telefone} em \Id{arquivo}}
          \Statep{\Write \Id{email} em \Id{arquivo}}
          \Statep{\Write \Id{mês} em \Id{arquivo}}
          \Statep{\Write \Id{ano} em \Id{arquivo}}
      \EndWhile
      \Statep{Encerre o acesso a \Id{arquivo}}
    \EndProcedure
\end{algorithmic}

\endgroup
\end{algorithm}

\end{tcolorbox}

\end{tcolorbox}

Function Description quarto.doc.add\_html\_dependency(dep) Add an HTML
dependency (additional resources and content) to a document. See docs on
the HTML Dependencies below for additional details.
quarto.doc.attach\_to\_dependency(name, attach) Attach a file to an
existing dependency. attach is a file path relative to the Lua filter or
table with \texttt{path} and \texttt{name} for renaming the file as its
copied. quarto.doc.use\_latex\_package(pkg, opt) Adds a
\texttt{\textbackslash{}usepackage} statement to the LaTeX output (along
an options string specified in opt)
quarto.doc.add\_format\_resource(path) Add a format resource to the
document. Format resources will be copied into the directory next to the
rendered output. This is useful, for example, if your format references
a bst or cls file which must be copied into the LaTeX output directory.

Function Description quarto.doc.add\_html\_dependency(dep) Add an HTML
dependency (additional resources and content) to a document. See docs on
the HTML Dependencies below for additional details.
quarto.doc.attach\_to\_dependency(name, attach) Attach a file to an
existing dependency. attach is a file path relative to the Lua filter or
table with \texttt{path} and \texttt{name} for renaming the file as its
copied. quarto.doc.use\_latex\_package(pkg, opt) Adds a
\texttt{\textbackslash{}usepackage} statement to the LaTeX output (along
an options string specified in opt)
quarto.doc.add\_format\_resource(path) Add a format resource to the
document. Format resources will be copied into the directory next to the
rendered output. This is useful, for example, if your format references
a bst or cls file which must be copied into the LaTeX output directory.

Function Description quarto.doc.add\_html\_dependency(dep) Add an HTML
dependency (additional resources and content) to a document. See docs on
the HTML Dependencies below for additional details.
quarto.doc.attach\_to\_dependency(name, attach) Attach a file to an
existing dependency. attach is a file path relative to the Lua filter or
table with \texttt{path} and \texttt{name} for renaming the file as its
copied. quarto.doc.use\_latex\_package(pkg, opt) Adds a
\texttt{\textbackslash{}usepackage} statement to the LaTeX output (along
an options string specified in opt)
quarto.doc.add\_format\_resource(path) Add a format resource to the
document. Format resources will be copied into the directory next to the
rendered output. This is useful, for example, if your format references
a bst or cls file which must be copied into the LaTeX output directory.

See Algoritmo~\ref{alg-blue} and Algoritmo~\ref{alg-red}.

\bookmarksetup{startatroot}

\chapter{Two}\label{two}

\begin{algorithm}[H]
\caption{\label{alg-red}}
\begingroup%

\begin{algorithmic}<language = brazilian, keyword color = red>
    \Description Criação de um arquivo com informações para manter uma agenda simplificada.
    \Input Uma sequência de contatos, separada nos campos \Id{nome}, \Id{email}, \Id{telefone}, \Id{mês} e \Id{ano}
    \Output Os dados de entrada armazenados em um arquivo
    \Procedure{CrieAgenda}{}[Criação a partir de dados externos]
      \Statep{Crie \Id{arquivo} como um arquivo vazio com acesso de escrita}
      \While{há dados para na entrada}
          \Statep{Obtenha os valores para \Id{nome}, \Id{email}, \Id{telefone}, \Id{mês} e \Id{ano}}
          \Statex
          \Statep{\Write \Id{nome} em \Id{arquivo}}
          \Statep{\Write \Id{telefone} em \Id{arquivo}}
          \Statep{\Write \Id{email} em \Id{arquivo}}
          \Statep{\Write \Id{mês} em \Id{arquivo}}
          \Statep{\Write \Id{ano} em \Id{arquivo}}
      \EndWhile
      \Statep{Encerre o acesso a \Id{arquivo}}
    \EndProcedure
\end{algorithmic}

\endgroup
\end{algorithm}

See Algoritmo~\ref{alg-blue} and Algoritmo~\ref{alg-red} and also
Algoritmo~\ref{alg-green}.

\bookmarksetup{startatroot}

\chapter{Three}\label{three}

\begin{algorithm}[H]
\caption{\label{alg-green}}
\begingroup%

\begin{algorithmic}<language = brazilian, keyword color = green!75!black>
    \Description Criação de um arquivo com informações para manter uma agenda simplificada.
    \Input Uma sequência de contatos, separada nos campos \Id{nome}, \Id{email}, \Id{telefone}, \Id{mês} e \Id{ano}
    \Output Os dados de entrada armazenados em um arquivo
    \Procedure{CrieAgenda}{}[Criação a partir de dados externos]
      \Statep{Crie \Id{arquivo} como um arquivo vazio com acesso de escrita}
      \While{há dados para na entrada}
          \Statep{Obtenha os valores para \Id{nome}, \Id{email}, \Id{telefone}, \Id{mês} e \Id{ano}}
          \Statex
          \Statep{\Write \Id{nome} em \Id{arquivo}}
          \Statep{\Write \Id{telefone} em \Id{arquivo}}
          \Statep{\Write \Id{email} em \Id{arquivo}}
          \Statep{\Write \Id{mês} em \Id{arquivo}}
          \Statep{\Write \Id{ano} em \Id{arquivo}}
      \EndWhile
      \Statep{Encerre o acesso a \Id{arquivo}}
    \EndProcedure
\end{algorithmic}

\endgroup
\end{algorithm}

See Algoritmo~\ref{alg-blue} and Algoritmo~\ref{alg-red}.


\backmatter

\end{document}
